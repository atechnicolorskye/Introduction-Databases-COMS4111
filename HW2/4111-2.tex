\documentclass[twoside]{Homework}

\studname{Si Kai Lee}
\uni{sl3950}
\studmail{sl3950@columbia.edu}
\coursename{Introduction to Databases}
\hwNo{2}

\begin{document}
\maketitle

\section*{Problem 2}
\subsection*{a}
Let liquor be $L$ and sales be $S$.\\
\noindent 
Let $\rho(C(1->\textrm{lid1}, 7->\textrm{lid2}), L \times S)$ be $X$.\\
\noindent
Names = $\pi_{\textrm{name}}(\sigma_{\substack{\textrm{lid1}\ =\ \textrm{lid2}\\ \textrm{month}\ =\ \textrm{December}\\ \textrm{county}\ =\ \textrm{Polk}\\\textrm{quantity}\ >=\ 1}}(X))
\cap 
\pi_{\textrm{name}}(\sigma_{\substack{\textrm{lid1}\ =\ \textrm{lid2}\\ \textrm{month}\ =\ \textrm{December}\\ \textrm{county}\ =\ \textrm{Linn}\\\textrm{quantity}\ >=\ 1}}(X))$

\subsection*{b}
Let liquor be $L$ and sales be $S$.\\
\noindent 
Let $L \bowtie S$ be $X$.\\
\noindent
Manufacturers = $\pi_{\textrm{manufacturer}}
\sigma_{1 \neq 9}((\sigma_{\substack{\textrm{month}\ =\ \textrm{January}\\ \textrm{county}\ =\ \textrm{Polk}\\\textrm{quantity}\ >=\ 1}}(X)
\bowtie_{\textrm{manufacturer}}
\sigma_{\substack{\textrm{month}\ =\ \textrm{January}\\ \textrm{county}\ =\ \textrm{Polk}\\\textrm{quantity}\ >=\ 1}}(X)))$

\section*{Problem 3}
\subsection*{a}
\begin{table}[h]
\begin{tabular}{ll}
B & D \\
x & c \\
y & a \\
x & a
\end{tabular}
\end{table}
\newpage
\subsection*{b}
\begin{table}[h]
    \begin{tabular}{llll}
    A & B & D & A \\
    1 & x & c & 1 \\
    3 & y & a & 1 \\
    3 & x & a & 1 \\
    1 & x & c & 3 \\
    3 & y & a & 3 \\
    3 & x & a & 3 \\
    \end{tabular}
\end{table}

\subsection*{c}
\begin{table}[h]
    \begin{tabular}{llllll}
    A & B & C & A & B & D\\
    1 & x & a & 3 & y & a\\
    1 & x & a & 3 & x & a\\
    \end{tabular}
\end{table}

\subsection*{d}
Since T1 as T1 - T2 = T1, hence T1- (T1 -T2) results in an empty table. 

\subsection*{e}
Since T1 does not have a column D, there would not be a common column to natural join, hence an empty table 
% http://stackoverflow.com/questions/18738836/relational-algebra-natural-join-few-basics
\end{document} 