\documentclass[twoside]{Homework}
\usepackage{wasysym}

\studname{Si Kai Lee}
\uni{sl3950}
\studmail{sl3950@columbia.edu}
\coursename{Introduction to Databases}
\hwNo{2}

\begin{document}
\maketitle

\section*{Problem 2}
\subsection*{a}
Let liquor be $L$ and sales be $S$.\\
\noindent 
Let $\rho(C(1->\textrm{lid1}, 7->\textrm{lid2}), L \times S)$ be $X$\\
\noindent
Names = $\pi_{\textrm{name}}(\sigma_{\substack{\textrm{lid1}\ =\ \textrm{lid2}\\ \textrm{month}\ =\ \textrm{December}\\ \textrm{county}\ =\ {Polk}\\\textrm{quantity}\ >=\ 1}}(X)
\cap 
\sigma_{\substack{\textrm{lid1}\ =\ \textrm{lid2}\\ \textrm{month}\ =\ \textrm{December}\\ \textrm{county}\ =\ {Linn}\\\textrm{quantity}\ >=\ 1}}(X))$

\subsection*{b}
Let liquor be $L$ and sales be $S$.\\
\noindent 
Let $\rho(C(1->\textrm{lid1}, 7->\textrm{lid2}), L \times S)$ be $X$\\
\noindent
Manufacturers = $\pi_{\textrm{manufacturer}}
\sigma_{1 \neq 10}((\sigma_{\substack{\textrm{lid1}\ =\ \textrm{lid2}\\ \textrm{month}\ =\ \textrm{January}\\ \textrm{county}\ =\ {Polk}\\\textrm{quantity}\ >=\ 1}}(X)
\bowtie_{\textrm{manufacturer}}
\sigma_{\substack{\textrm{lid1}\ =\ \textrm{lid2}\\ \textrm{month}\ =\ \textrm{January}\\ \textrm{county}\ =\ {Polk}\\\textrm{quantity}\ >=\ 1}}(X)))$

\section*{Problem 3}
\subsection*{a}
\begin{table}[h]
\begin{tabular}{ll}
B & D \\
x & c \\
y & a \\
x & a
\end{tabular}
\end{table}
\newpage
\subsection*{b}
\begin{table}[h]
    \begin{tabular}{llll}
    A & B & D & A \\
    1 & x & c & 1 \\
    3 & y & a & 1 \\
    3 & x & a & 1 \\
    1 & x & c & 3 \\
    3 & y & a & 3 \\
    3 & x & a & 3 \\
    1 & x & c & 3 \\
    3 & y & a & 3 \\
    3 & x & a & 3 \\
    \end{tabular}
\end{table}

\subsection*{c}
\begin{table}[h]
    \begin{tabular}{llllll}
    A & B & C & A & B & D \\
    1 & x & a & 3 & y & c \\
    1 & x & a & 3 & x & a \\
    \end{tabular}
\end{table}

\subsection*{d}
Empty T1 as T1 - T2 = T1, hence T1- (T1 -T2) = Null 
\begin{table}[h]
\begin{tabular}{lll}
A & B & C\\
Null & Null & Null\\
Null & Null & Null\\
Null & Null & Null
\end{tabular}
\end{table}


\newpage
\subsection*{e}
Since the two tables have no common attributes, cross product with empty T2
\begin{table}[h]
    \begin{tabular}{llllll}
    A & B & C & A & B & D \\
    1 & x & a &  &  &  \\
    2 & y & b &  &  &  \\
    3 & z & b &  &  &  \\
    1 & x & a &  &  &  \\
    2 & y & b &  &  &  \\
    3 & z & b &  &  &  \\
    1 & x & a &  &  &  \\
    2 & y & b &  &  &  \\
    3 & z & b &  &  &  \\
    \end{tabular}
\end{table}

\end{document} 